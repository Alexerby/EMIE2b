%%%%%%%%%%%%%%%%%%%%%%%%%%%%%%%%%%%%%%%%%%%%%%%%%%%%%
%                   PACKAGES                        %
%%%%%%%%%%%%%%%%%%%%%%%%%%%%%%%%%%%%%%%%%%%%%%%%%%%%%
\documentclass{article}
\usepackage{graphicx} % Required for inserting images
\usepackage{mdframed}
\usepackage{xcolor}
\usepackage{lipsum}
\usepackage[default]{opensans}
\usepackage{setspace}
\usepackage{multicol}
\usepackage[margin=1in]{geometry} % Adjust the margins as needed
\usepackage{parskip}
\usepackage{sectsty}
\usepackage{colortbl}
\usepackage{array}
\usepackage{adjustbox}
\usepackage{wrapfig}

%%%%%%%%%%%%%%%%%%%%%%%%%%%%%%%%%%%%%%%%%%%%%%%%%%%%%
%                   STYLES                          % 
%%%%%%%%%%%%%%%%%%%%%%%%%%%%%%%%%%%%%%%%%%%%%%%%%%%%%

\definecolor{mydarkblue}{RGB}{19, 52, 85}

\mdfdefinestyle{titlestyle}{
    backgroundcolor=mydarkblue, 
    fontcolor=white,
    leftmargin=-80,
    rightmargin=-80
}

\mdfdefinestyle{sectionstyle}{
    backgroundcolor=mydarkblue,
    fontcolor=white,
    leftmargin=0,
    rightmargin=0,
    linewidth=0pt,
    linecolor=mydarkblue,
}
 
\setstretch{1.25}
\sectionfont{\color{mydarkblue}}  % sets colour of sections
\subsectionfont{\color{mydarkblue}}  % sets colour of subsection

\makeatletter
\renewcommand{\paragraph}{\@startsection{paragraph}{4}{\z@}{3.25ex \@plus 1ex \@minus .2ex}{-1em}{\color{mydarkblue}\normalfont\small\bfseries}}
\makeatother



\let\oldtextit\textit
\renewcommand{\textit}[1]{\textcolor{mydarkblue}{\oldtextit{#1}}}

%%%%%%%%%%%%%%%%%%%%%%%%%%%%%%%%%%%%%%%%%%%%%%%%%%%%%
%                   ENVIRONMENTS                    % 
%%%%%%%%%%%%%%%%%%%%%%%%%%%%%%%%%%%%%%%%%%%%%%%%%%%%%


%%%%%%%%%%%%%%%%%%%%%%%%%%%%%%%%%%%%%%%%%%%%%%%%%%%%%%
% Change padding and margin for \maketitle
\setlength{\topmargin}{-1in}         % Adjust the top margin
\setlength{\oddsidemargin}{0in}     % Adjust the left margin for odd-numbered pages
\setlength{\evensidemargin}{0in}    % Adjust the left margin for even-numbered pages
\setlength{\textwidth}{6.5in}       % Adjust the text width


\makeatletter
\renewcommand{\maketitle}{%
  \begin{mdframed}[style=titlestyle]
    \vspace*{0.5in}
    \hspace*{\dimexpr-\leftmargin-\oddsidemargin--1in}\raggedright % Positioning title
    \textbf{\LARGE\@title} \\
    \hspace*{\dimexpr-\leftmargin-\oddsidemargin--1in}\raggedright % Positioning text under title
     An analysis of the relationship between inflation and money supply in Sweden \\
    \vspace*{0.5in}
  \end{mdframed}
  \thispagestyle{empty}}
\makeatother

\newenvironment{myenv}
{
    \color{mydarkblue} % Set font color to medium grey
    \fontsize{9pt}{9pt}\selectfont % Set font size to 10pt

}
{
    % Optional code for the end of the environment
}

\renewcommand{\arraystretch}{1.25} 




\title{Inflationary Tendencies of Increases in M1}
\date{}

\begin{document}
\textcolor{mydarkblue}{\textbf{Course: }EC2408} \\
\textcolor{mydarkblue}{\textbf{Author}: Alexander Eriksson Byström (20020228-0293)}
\hspace*{5cm}\textcolor{mydarkblue}{\textbf{\today}} 

    \maketitle

    \vspace*{1cm}
    \section*{Introduction: In the face of economic adversity}
    As of writing this paper the American Congress is dealing with another debt crisis where the Joe
    Biden Administrationis negotiating with the speaker of the House of Represenatatives Kevin Mc-
    Carthy about raising the national debt ceiling of \$31.4T in order to not default on its current debt.
    In reality, the United States hit its debt ceiling already, in January this year, but have been able to
    move assets and provide enough liquidity to not default. However, the method of shuffling existing
    resources is not sustainable in the long run and if the administration and the Republicans do not
    strike a deal by next week we will find ourselves in a financial crisis making the economic impact of
    theCorona-pandemic seem like a breeze in comparison.

    When the Standard \& Poors 500 indice dove 35.92\% in 32 business days Fed had no choice but
    to react, and the response involved implementing quantitative easing and balance sheet operations,
    which effectively expanded the money supply 5 times over the following twelve months.

    In Sweden, the monetary policy of the Riksbank have not been as expansive as the policy of Fed.
    However, Corona naturally had an immense impact and an expansive monetary policy was also
    here unevitable. Some of the monetary policies here included loans of 500BSEK to swedish banks,
    purchasing of securities of up to 300BSEK and in addition to that offering of loans in USD.2 Despite
    the relatively moderate monetary policies implemented by the Riksbank, there was a notable increase
    in the money supply. This occurred during the period of persistently low interest rates that prevailed
    during 2020. Commercial banks found it advantageous to borrow from the Riksbank to effectively
    utilize the fractional reserve banking system, which in turn had an impact on the M1 money supply.

    This paper aims to examine the conventional economic dogma that asserts a positive relationship
    between the money supply and inflationary pressure. To accomplish this, vector autoregressive
    models and impulse response functions will be employed for analysis.

    \newpage

    \vspace*{1cm}
    \section*{Data}
    We have in the study selected monthly data from Statistics Swedens ”Statistikdatabasen” for money
    supply measured in M1 and the annual change in consumer price index (i.e. inflation). The data
    span is limited to between January 2004 to November 2022, giving us 232 observations. The chosen
    time limitation is not arbitrary but is based on the available data. In this case, the first
    observation for the M1 variable is in January 2004, while the last observation for both our variables
    is in April 2023, as obtained from Statistics Sweden’s ”Statistikdatabas.” By using this time range,
    the aim is to include as much of the variance in the dataset as possible, considering the available
    data points.

    \section*{Empirical Methodology and Obtained Results}
    
    \subsection*{Exploring Stationarity and Cointegration: Analyzing Time Series Properties}

        \paragraph*{Rationale for Incorporation of a Trend Model.}
        Before constructing our model, we need to evaluate what type of model we are going 
        to use throughout our analysis. I am here going to argue that in the context of inflation and money 
        supply both variables have a deterministic trend component, this due to two factors. Firstly inflation
        is not considered a stochastic process since changes in other economic variables such as changes
        in monetary policy, supply and demand dynamics, and the fiscal policies have an impact on the change in 
        prices. Secondly, since M1 represents the most liquid form of money, this money will reasonably be 
        highly influenced by the Riksbanks monetary policy, fiscal policy and the underlying economic condition. 
        \footnote{This is not to say that other types of money beside Fiat Money would be less influenced. 
        Nevertheless, it is stating that M1 is highly impacted but the mentioned policy and conditions.}
        Therefore, I argue that this variable should not be considered stochastic.\footnote{We can also see
        that our variables, particularly M1, does exhibit a pattern of trend when looking at the decomposition
        of the timeseries (figure 1.1 and 1.2), advancing the proposition of using a model with trend.}

        \paragraph*{Augmented Dickey Fuller test.} The first test which we need to subject our model to 
        is the ADF test. Here we test the null ($H_0$) of a prevalent unit root. This test is a one-sided 
        test conducted on the negative side. We will be focusing on the $\tau_3$ statistics due to the
        incorporation of a coefficient and trend in accordance with the argument in the paragraph above.
        The conclusion from this test is that we are from now on dealing with two models integrated of the 
        first order ($I(1)$).\footnote{See Table 1.1 \& 1.2.}
        
        \paragraph*{Cointegration.}
        Due to us now dealing with two models integrated of the first order we need to test whether there 
        exists any long-term equilibrium relationship between inflation and M1. More specifically, we need 
        to construct a linear combination of our integrated variables and perform an Engle-Granger test 
        which is a test where we regress our error correction term on its lagged values and lagged 
        differences. I am here going to argue for using money supply as the dependent variable in our 
        linear regression as money supply in conformity with traditional economics would be a driving 
        cause of inflation (if M1 outpaces economic growth). The second step in our EG-ADF test is the ADF 
        test itself. The outcome of the EG-ADF test, when looking at $\tau_2$ (intercept, no trend) is that 
        we find a value of -1.57 which implies that we can not reject the null of no cointegration when comparing 
        it to the critical value for the EG-ADF test of -3.41.\footnote{For full table of critical values for 
        EG-ADF statistic, see Stock, James \& Watson Mark (2019). Introduction to Econometrics, p. 665.} 
        Therefore, based on our dataset, Engle-Granger ADF test does not appear to provide sufficient 
        evidence of cointegration between M1 and inflation.

    
    \subsection*{Model Selection and VAR Modeling: Finding the Best Fit}
        \paragraph*{Varselect (selecting best lag for model)}
        \paragraph*{Establishing the model (var result)}

    \subsection*{Assessing Model Validity: Diagnostic Testing for VAR Analysis}
        \paragraph*{Residual Analysis} is performed using the \textit{autocorrelation function} and 
        \textit{Ljung-Box test}.The autocorrelation function shows us a strong autocorrelation in as well
        inflation as in M1 supply. The autocorrelation in inflation, in contrast, exhibits a considerably 
        more rapid decrease compared to the autocorrelation in M1. A significant autocorrelation in inflation 
        is exhibited for approximately one year, where as the autocorrelation in M1 is shown to last much longer,
        approximately 5 years.\footnote{Figure not in included.} When performing a Ljung-Box test we can confirm that 
        the model does exhibit autocorrelation with a p-value of 1.182e-04, which is noticeably below our significance 
        level used throughout this analysis of 5\%.

        \paragraph*{Model Fit and Assumptions} must be confirmed using a Jarque-Bera test. Here we examine whether 
        the model is exercising a normal distribution ($H_0: JB = 0$), or if we can see that our variables are exercising
        a distribution which is not normally distributed.



    \newpage

    %%%%%%%%%%%%%%%%%%%%%%%%%%%%%%%%
    % Stationarity testing         % 
    %%%%%%%%%%%%%%%%%%%%%%%%%%%%%%%%

    % Decomposition graphs
    \begin{mdframed}[style=sectionstyle]
        Decomposition of Inflation and Money Supply
    \end{mdframed}
    \begin{figure}[htbp]
        \includegraphics*[width=0.5\textwidth]{../graphs/Decomposition of Inflation.png}%
        \includegraphics*[width=0.5\textwidth]{../graphs/Decomposition of M1.png}
        \vspace{-\baselineskip}
        \begin{myenv}
            \hspace*{0.1\textwidth}
            Figure 1.1: Decomposition of Inflation \hspace*{0.15\textwidth} 
            Figure 1.2: Decomposition of M1
        \end{myenv}
    \end{figure}
    \vspace*{0.1cm}
    \begin{mdframed}[style=sectionstyle]
        Augmented Dickey Fuller: testing for stationarity
    \end{mdframed}
        
    % Tabular: CPI & CPI (first-order-difference)
    \begin{adjustbox}{width=\textwidth}
        \tiny % Set the font size to 10pt
            \begin{tabular}{lrrrrr}
                \rowcolor{mydarkblue}
                \textcolor{white}{Test Statistic}     & \textcolor{white}{$I(0)$}        & \textcolor{white}{$I(1)$}            & \textcolor{white}{1pct}           & \textcolor{white}{5pct}     & \textcolor{white}{10pct}                   \\
                $\tau_3$                              &-1.56                                  &-6.09                                      &-3.99                              &-3.43                        &-3.13                       \\
                $\phi_3$                              & 1.81                                  & 18.57                                     & 8.43                              & 6.49                        & 5.47                       \\
                $\tau_2$                              &-1.15                                  &-5.97                                      &-3.46                              &-2.88                        &-2.57                       \\
                $\phi_1$                              & 1.04                                  & 17.84                                     & 6.52                              & 4.63                        & 3.81                       \\
                $\tau_1$                              &-0.41                                  &-5.93                                      &-2.58                              &-1.95                        &-1.62                       \\
            \end{tabular}
            \hspace*{0.5cm}

            
            \begin{tabular}{lrrrrr}
                \rowcolor{mydarkblue}
                \textcolor{white}{Test Statistic}     & \textcolor{white}{$I(0)$}        & \textcolor{white}{$I(1)$}            & \textcolor{white}{1pct}           & \textcolor{white}{5pct}     & \textcolor{white}{10pct}             \\
                $\tau_3$                              &-1.60                                  &-5.61                                      &-3.99                              &-3.43                        &-3.13                                 \\
                $\phi_3$                              & 1.78                                  & 15.94                                     & 8.43                              & 6.49                        & 5.47                                 \\
                $\tau_2$                              & 0.41                                  &-5.56                                      &-3.46                              &-2.88                        &-2.57                                 \\
                $\phi_1$                              & 4.28                                  &15.48                                      & 6.52                              & 4.63                        & 3.81                                 \\
                $\tau_1$                              & 2.67                                  &-4.68                                      &-2.58                              &-1.95                        &-1.62                                 \\
            \end{tabular}
            \vspace*{0.5cm}
    \end{adjustbox}
    \begin{myenv}
        \hspace*{0.1\textwidth}
        Table 1.1: Augmented Dickey Fuller CPI
        \hspace*{0.15\textwidth}
        Table 1.2: Augmented Dickey Fuller M1
    \end{myenv}
            

    
\end{document}